\chapter{Finding driver mutations} \label{chap:finding_driver_mutations}

Although the true explanation for mutual exclusivity remains unknown, and its therapeutic potential is still uncertain, this phenomenon is frequently observed in data and is thought to potentially lead to discoveries in cancer treatment. Existing approaches can be categorized into two types: \textbf{\textit{de novo}} approaches, which identify mutually exclusive patterns using only genomic data from patients, and \textbf{\textit{knowledge-based}} methods, which integrate the analysis with external \textit{a priori} information \cite{survey}. \textit{De novo} approaches might lack sufficient information as they do not utilize existing databases. Conversely, given that our understanding of gene and protein interactions in humans is still incomplete and many pathway databases fail to accurately represent the specific pathways and interactions present in cancer cells, \textit{knowledge-based} approaches may be limited by their dependence on existing data sources. Consequently, \textit{de novo} methods might yield new but potentially less accurate results, while \textit{knowledge-based} approaches may limit the discovery of novel biological insights \cite{multi-dendrix}.

\section{Approaches}

\subsection{Dendrix}

placeholder. \todo{nel capitolo precedente l'ho menzionato, cosa dovrei fare? parlarne? io non l'ho analizzato perché parla di Catene di Markov e Monte Carlo (MCMC), lascio perdere e non faccio una sezione per lui?}

\subsection{Multi-Dendrix}

Multi-Dendrix \cite{multi-dendrix} trys to solve Dendrix's same problem, described in \cref{mwsp}. \todo{qui manca una sezione iniziale in cui multi-dendrix descrive i motivi per cui l'approccio greedy di dendrix potrebbe non trovare soluzioni ottimali, e perché il loro approccio che trova le soluzioni tutte in un colpo solo è migliore, ma per ora non lo inserisco perché non so come trattare dendrix}

For these reasons, Multi-Dendrix's authors formulate Dendrix's problem as an \textbf{ILP} (\textit{Integer Linear Program}), which they refer to as $\mathrm{Dendrix}_{\mathrm{\textit{ILP}}}(k)$. Consider a gene set $M$ defined by a set of indicator variables, one for each gene $j \in M$ as follows

\begin{equation}
    I_M(j) = 1 \iff j \in M
\end{equation}

and a set of indicator variables, one for each patient $i$, described like this

\begin{equation} \label{c_idefn}
    C_i(M) = 1 \iff \exists g \in M \mid i \in \Gamma(g)
\end{equation}

\begin{definition}[$\mathrm{Dendrix}_{\mathrm{\textit{ILP}}}(k)$]
    $\mathrm{Dendrix}_{\mathrm{\textit{ILP}}}(k)$ is defined by the following ILP:

    \begin{equation} \label{weight_dendrix_ilp}
        \mathrm{maximize} \sum_{i = 1}^m {\rbk{2 \cdot C_i(M) - \sum_{j = 1}^n I_M(j) \cdot a_{i, j}}}
    \end{equation}

    \begin{equation} \label{second_constr_dendrix_ilp}
        \mathrm{subject \ to} \sum_{j = 1}^n{I_M(j) = k},
    \end{equation}

    \begin{equation} \label{third_constr_dendrix_ilp}
        \sum_{j = 1}^n I_M(j) \cdot {a_{i, j}} \ge C_i(M),
    \end{equation}

    \begin{equation*}
        \mathrm{for\ } 1 \le i \le m
    \end{equation*}
\end{definition}

Note that the sum in \cref{weight_dendrix_ilp} the second version of the definition provided in \cref{weight}.

\begin{lemma}[Correctness of $\mathrm{Dendrix}_{\mathrm{\textit{ILP}}}(k)$] Given a gene set $M$, the sum in \cref{weight_dendrix_ilp} correctly evaluates $W(M)$.
\end{lemma}

\begin{proof}
    Rearranging the terms in \cref{weight_dendrix_ilp} $$\sum_{i = 1}^m {\rbk{2 \cdot C_i(M) - \sum_{j = 1}^n I_M(j) \cdot a_{i, j}}} = 2\sum_{i = 1}^m {C_i(M)} - \sum_{i = 1}^m {\sum_{j = 1}^n {I_M(j) \cdot a_{i, j}}}$$ and it is trivial to check that $$\abs{\Gamma(M)} = \sum_{i = 1}^m {C_i(M)}$$ by definition, and $$\sum_{g \in M}{\abs{\Gamma(g)}} = \sum_{i = 1}^m {\sum_{j = 1}^n {I_M(j) \cdot a_{i, j}}}$$ because the RHS counts the number of cells of $A$ such that $a_{i, j} = 1$ for every $j \in M$.
\end{proof}

\cref{second_constr_dendrix_ilp} limits the size of $M$ to be exactly $k$; moreover, note that \cref{third_constr_dendrix_ilp} only forces $C_i(M) = 0$ when the $i$-th patient has no mutated genes in $M$, but does not force $C_i(M) = 1$ when the patient has at least one, as required by \cref{c_idefn}. However, the objective function will be maximized when $C_i(M)=1$ thus \cref{c_idefn} is satisfied.

placeholder. \todo{qui manca un pezzo in cui si spiega che multi-dendrix}

As outlined in \cref{multi_dendrix_2nd_chap}, Multi-Dendrix proposes that the most effective approach to conducting the research is to identify a collection of gene sets that maximizes the sum of their individual weights. To achieve this result, they solve the following problem, which is an extension of \cref{mwsp}:

\begin{displayquote}
    \textbf{Multiple Maximum Weight Submatrices Problem}: Given an $m \times n$ mutation matrix $A$, and integer $t > 0$, find a collection $M = \{M_1, \ldots, M_t\}$ of $m \times k$ \todo{$k$ DOES NOT APPEAR ANYWHERE IN THE ILP???????} column submatrices that maximizes

    \begin{equation}
        W'(M) := \sum_{\rho = 1}^t {W(M_\rho)}
    \end{equation}
\end{displayquote}

Note that this problem is NP-Hard, as for the case $t = 1$. Additionally, while $W'(M)$ increases with greater mutual exclusivity among mutations \textit{within} a gene set, there is no constraint on mutations \textit{between} different gene sets. Furthermore, collections $M$ with a large value of $W'(M)$ are also likely to have higher coverage $\Gamma(M_\rho)$ for each individual gene set $\rho$. As a result, optimal solutions tend to produce collections where many patients have mutations in more than one gene set, or they may be pairs or larger groups of co-occurring mutations, a phenomenon observed in cancer. \todo{qua inseriscono una citazione, potrebbe valere la pena di inserirla e/o indagare?}

\begin{definition}[Multi-Dendrix]
    $\mathrm{Multi}$-$\mathrm{dendrix}$ is defined by the following ILP:

    \begin{equation} \label{weight_dendrix_ilp}
        \mathrm{maximize} \sum_{\rho = 1}^t {\sum_{i = 1}^m {\rbk{2 \cdot C_i(M) - \sum_{j = 1}^n I_M(j) \cdot a_{i, j}}}}
    \end{equation}

    \begin{equation} \label{second_constr_dendrix_ilp}
        \mathrm{subject \ to} \sum_{j = 1}^n I_M(j) \cdot {a_{i, j}} \ge C_i(M),
    \end{equation}

    \begin{equation*}
        \mathrm{for\ } 1 \le i \le m, \ 1 \le \rho \le t,
    \end{equation*}

    \begin{equation}
        \sum_{\rho = 1}^t{I_{M_\rho}(j)} \le 1, \ 1 \le j \le m,
    \end{equation}

    \begin{equation}
        k_\mathrm{min} \le \sum_{j = 1}^n {I_M(j)} \le k_\mathrm{max}.
    \end{equation}
\end{definition}

placeholder. \todo{spiega le formule}

placeholder. \todo{ultimo paragrafo}

\subsection{MDPFinder}

\subsection{Mutex}

\subsection{C3}

\cleardoublepage
