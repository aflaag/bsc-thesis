
\chapter{TODO} \label{chap:TODO}

\section{Multi-Dendrix}

\subsection{Dendrix}

A very well known paper, which developed two algorithms called \curlyquotes{Dendrix} \cite{dendrix}, gave the following mathematical formalization to the properties of \textbf{mutual exclusivity} and \textbf{coverage}. Consider a so called \curlyquotes{mutation matrix} $A$, with $m$ rows and $n$ columns, where each row represents a patient and each column represents a gene; the entry $a_{i, j}$ is equal to 1 if and only if gene $j$ is mutated in patient $i$ \todo{add table as example?}.



Given a gene $g$, let

\begin{equation}
    \Gamma(g) = \{i : a_{i, g} = 1\}
\end{equation}

denote the set of patients which have $g$ mutated; futhermore, given a set of $M$ genes, let the \textbf{coverage} be

\begin{equation}
    \Gamma(M) = \bigcup_{g \in M}{\Gamma(g)}
\end{equation}

which denotes the set of patients in which at least one of the genes in $M$ is mutated. In accordance with the previous definitions of mutual exclusivity, we say that a set $M$ of genes is \textbf{mutually exclusive} \textit{if no patient has more than one mutated gene}, formally

\begin{equation}
    \forall g, g' \in M \quad \Gamma(g) \cap \Gamma(g') = \varnothing
\end{equation}

Any gene set can be thought as a $m \times k$ submatrix of a mutation matrix $A$, up to rearranging $A$'s columns --- their order does not matter since they represent genes. Accordingly, such a submatrix is said to be \textbf{mutually exclusive} \textit{if each row contains at most one 1}.

Furhermore, given a gene set $M$, the following properties are formalized:

\begin{enumerate}[label=\roman*), font=\itshape]
    \item \textit{coverage}: most patients have at least one mutation in $M$;
    \item \textit{approximate exclusivity}: most patients have exactly one mutation in $M$.
\end{enumerate}

To measure these two attributes, a measure what quantifies the trade-off between coverage and mutual exclusivity is introduced. Given a set $M$ of genes, the \textit{coverage overlap} is defined as follows:

\begin{equation}\label{cov_over}
    \omega(M) = \sum_{g \in M}{\abs{\Gamma(g)}} - \abs{\Gamma(M)}
\end{equation}

Note that the sum in \cref{cov_over} is the number of 1s in $M$'s corresponding submatrix; hence, the 

\cleardoublepage
