\chapter*{Conclusions}

\addcontentsline{toc}{chapter}{Conclusions}

This work explored the challenges of finding effective targets for targeted therapies, specifically driver mutations, and why driver pathways offer a valuable strategy for identifying key mutations in cancer development. In recent years, efforts to identify cancer driver pathways have revealed various biological phenomena in genomic data, such as mutual exclusivity and the coverage of driver genes.

This work presented several studies that developed various metrics, each based on distinct biological assumptions, to assess mutual exclusivity and other biological phenomena, and described how these measures were incorporated into their respective algorithms. ILP, clustering, greedy, MCMC, and genetic algorithms were presented, with the latter likely being the most promising in cancer research due to the infrequent occurrence of exact mutual exclusivity and coverage in real cancer genomic data.

Future research may integrate emerging technologies, such as single-cell sequencing, to further refine our understanding of driver pathways. Additionally, extending these algorithms to account for tumor heterogeneity and adaptive resistance mechanisms will be critical in expanding their applicability to diverse cancer types.

The identification of driver pathways and the ability to measure key genomic phenomena hold significant promise for improving the precision of targeted therapies. These advances could lead to more effective treatment strategies by focusing on the most critical mutations that drive cancer progression. Advancements in the understanding of cancer driver mutations bring the field closer to achieving personalized, more effective cancer treatments tailored to the unique genetic profiles of each patient.
