\chapter*{Conclusions}

\addcontentsline{toc}{chapter}{Conclusions}

In the first part of this work, an \textbf{overview} of cancer was provided, including its causes and the various types of treatments currently in use, with a particular emphasis on \textbf{targeted therapies}. Specifically, \textit{targeted therapies} show promise because they help reduce the \textbf{side effects} that characterize all existing cancer treatments. The primary goal of these therapies is to directly \textit{target driver mutations}, which play a crucial role in cancer development.

However, distinguishing between \textit{driver and passenger mutations} is very difficult, as both types of mutations can occur simultaneously, and passenger mutations often outnumber driver ones, making it challenging to identify which mutations \textit{significantly contribute} to cancer progression. Instead, identifying \textbf{driver pathways} appears to be a promising approach, as multiple driver mutations in different genes can produce \textit{similar downstream effects}, potentially causing various mutations to impact the same pathway across multiple samples.

Indeed, the second part of this work \textit{introduced multiple studies} that aimed to develop approaches for \textit{identifying driver pathways}. Specifically, the focus of this section was on the \textbf{biological assumptions} the various studies made, and how their developed \textit{metrics} reflected those assumptions. Notably, \textbf{mutual exclusivity} among mutations within a driver pathway is one of the most frequently observed \textit{phenomena} in genomic data, and many recent studies have formalized this biological characteristic in various ways.

Finally, the last part of this work examined the various \textbf{algorithms} employed by the studies introduced in the second part, illustrating how each study utilized its own developed \textit{measure} of \textbf{mutual exclusivity} and other biological assumptions in order to \textit{identify driver pathways}, and potentially discover new ones.

As a final note, \textbf{future research} may integrate \textit{emerging technologies}, such as \href{https://en.wikipedia.org/wiki/Single-cell_sequencing}{single-cell sequencing}, to further refine our understanding of \textbf{driver pathways}. Additionally, extending these algorithms to account for \href{https://en.wikipedia.org/wiki/Tumour_heterogeneity}{tumor heterogeneity} and \textit{adaptive resistance mechanisms} will be critical in expanding their applicability to diverse cancer types. The identification of \textbf{driver pathways} and the ability to measure key genomic phenomena hold significant promise for improving the precision of \textbf{targeted therapies}, which could lead to more powerful and personalized treatment, tailored to the unique genetic profiles of each patient.
