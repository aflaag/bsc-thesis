\chapter{Introduction} \label{chap:introduction}

\section{Context}

\subsection{Cancer}

Cancer is a medical condition characterized by an uncontrolled proliferation of the cells, allowing their infiltration into organs and tissue, altering their functions and structure. The exponential growth is caused by mutations in the cellular DNA, because it contains information that describes how cells should develop and multiply, and errors in these instructions may lead the cell to become cancerous. In the vast majority of cancer types one single aberration is not sufficient for cancer development and multiple mutations are necessary, some already present at birth like in cellular DNA, and others obtained throughout life either by chance or by lifestyle choices. Moreover, for the tumor to proliferate mutations on genes that regulate cell growth are needed \cite{Vogelstein2004}, in particular \textbf{proto-oncogenes} which promote mitosis, and \textbf{tumor suppressor genes} which discourage cell growth --- a process called \textbf{oncoevolution}. \todo{talk about oncogenes in depth?}

\subsection{Cancer treatment}

Research towards finding a cure for cancer is in constant development, because of the tumor's lethality and complexity. At present, the following are the main techniques used to remove, control, restrain and delay the effects of cancer \cite{cancer_treat}:

\begin{itemize}
    \item \textbf{surgery}, which allows the removal of the cancerous region, and it's usually reserved for solid tumors;
    \item \textbf{radiotherapy}, which uses x-rays to destroy tumor cells, and it's aimed towards the cancerous region as much as possible to preserve healthy cells; radiotherapy can increase the risk of developing other tumors because of x-rays, such as leukemia or sarcomas, and it can lead to delayed effects like dementia, amnesia or progressively worsening cognitive difficulties;
    \item \textbf{chemotherapy}, which blocks cellular division --- through cytotoxic drugs --- of both cancerous and healthy cells, inducing side effects to every rapidly renewing tissue;
    \item \textbf{hormone therapy}, through which the balance of specific hormones gets altered, which can lead to some side effects such as joint pain or osteoporosis;
    \item \textbf{target therapy}, i.e. drugs containing antibodies or inhibitory substances that target the cancer cell and promote its destruction by the immunity system; target therapy may be difficult to develop depending on the structure or the function of the target, it may induce unwanted side effects to various organs, and cancer cells can become increasingly resistant to this type of therapy if they find a way to develop which does not depend on the therapy's target \cite{target_therapy1} \todo{\href{https://www.cancer.org/cancer/managing-cancer/treatment-types/targeted-therapy/what-is.html}{check this out}}.
\end{itemize}

In particular, in recent years target therapy has been the subject of lots of research, because it could be a way to affect only the desired target, helping to reduce the side effects which currently characterize every available cancer cure, possibly limiting the damage to healthy cells \cite{target_therapy3}. \todo{expand target therapy on how it works? if yes, make subsection}

\section{Mutations}

\subsection{Cell signaling and signaling pathways}

\textbf{Cell signaling} is the process through which cells interact with themselves, other cells, or their environment; through cell signaling, signals are transduced, and they can be of many types, usually chemical but also pressure, voltage, temperature, or light signals \cite{cell_signaling}. \textbf{Pathways} are a series of actions between molecules inside a cell which lead to a change in the cell or the creation of some product \cite{pathway}. Pathways have a \textit{direction} in which the actions happen, and the terms \textit{upstream} and \textit{downstream} are used to indicate what happens at their beginning or their end respectively. For cancer study, of particular interest are \textbf{signaling pathways}, which allow the transduction of cell signals, because locating and blocking pathways that are responsible for the core functions of cancer growth could terminate the development of the latter. \todo{\href{https://www.ncbi.nlm.nih.gov/pmc/articles/PMC8002322/}{check this out}, also check if what i wrote is actually true, i think i read it somewhere but can't find the source right now; expand on cell signaling? expand of pathways? if yes, make subsections}

\subsection{Passenger and driver mutations}

There are two types of mutations in cancer: \textbf{passenger} and \textbf{driver} mutations. Passenger mutations don't provide direct benefits to tumor growth and development, while driver mutations can directly influence cancer, providing evolutionary advantage and allowing an increase in the number of tumor cells. A \textbf{driver gene} is a gene that contains at least one driver mutation, but it can also contain passenger mutations \todo{\href{https://www.aiom.it/wp-content/uploads/2019/02/20190524RM_21_Tommasi.pdf}{DO I ADD THIS A CITATION?}}; a \textbf{driver pathway} is made up of at least one driver gene. Driver mutations, genes, and pathways are of great scientific interest since hold an important role in cancer proliferation.

Driver genes can be classified into 12 signaling pathways, which regulate functions of survival, fate, and genomic maintenance of the cell \todo{use (and expand) this? same source as prev}.

\subsection{Classifying mutations}

To classify mutations into the two described categories, it is necessary to check their biological function, to this day a difficult task to accomplish. There are lots of methods that allow predicting the functional impact of mutations through \textit{a priori} knowledge, but usually, these approaches are not able to integrate information adequately across the various types of mutations, and because they are based on known information about already discovered proteins they are less successful for less studied ones. Thanks to the decreasing cost of DNA sequencing, it is now possible to categorize mutations by studying their frequency, i.e. driver mutations are proportionately the most recurrent in patients' genomes. However, this approach often fails because driver mutations vary between cancer patients' samples, even those having the same cancer type, significantly reducing the statistical potential. \cite{multi-dendrix}. It was found that there is little overlap between mutated genes over sample pairs even taken from the same patient \cite{mdpfinder}. This heterogeneity is mainly explained by the fact that driver mutations are mainly found in genes that are part of cell signaling pathways, thus different patients may harbor mutations in different pathway loci. Consequentially, it is not sufficient to look into single gene frequency, but it must be tested whether group genes are recurrently mutated. Moreover, the study must be done at the pathway level, because it is well known that there may be different mutations inside the same pathway, across multiple samples \cite{multi-dendrix}.

\subsection{Mutual exclusivity}

\subsection{\textit{De novo} and \textit{knowledge-based} approaches}

\cleardoublepage

