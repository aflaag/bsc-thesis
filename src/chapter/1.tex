\chapter{Introduction} \label{chap:introduction}

\section{Context}

\subsection{Cancer}

Cancer is a medical condition characterized by an uncontrolled proliferation of the cells, allowing their infiltration in organs and tissue, altering their functions and structure. The exponential growth is caused by cellular DNA mutations, because it contains information that describe how cells should develop and multiply, and errors in these instructions may lead the cell to become cancerous. In the vast majority of cancer type one single aberration is not sufficient to cancer development and multiple mutations are necessary, some already present at birth like in cellular DNA, others obtained throughout either by chance or by lifestyle choices. Moreover, in order for tumor to proliferate mutations on genes that regulate cell growth are needed \cite{Vogelstein2004}, in particular \textbf{proto-oncogenes} which promote mitosis, and \textbf{tumor suppressor genes} which discourage cell growth --- a process called \textbf{oncoevolution}. \todo{talk about oncogenes in depth?}

\subsection{Cancer treatment}

Research towards finding a cure for cancer is in constant development, because of tumor's lethality and complexity. At present, the following are the main techniques used to remove, control, restrain and delay the effects of cancer \cite{cancer_treat}:

\begin{itemize}
    \item \textbf{surgery}, which allows to phisically remove the cancerous region, and it's usually reserved to solid tumors;
    \item \textbf{radiotherapy}, which uses x-rays to destroy tumor cells, and it's aimed towards the cancerous region as much as possible to preserve healthy cells; radiotherapy can inrcease the risk of developing other tumors becaus of x-rays, such as leukimia or sarcomas, and it can lead to delayed effects like dementia, amnesia or progessively worsening cognitive difficoulties;
    \item \textbf{chemotherapy}, which blocks cellular division through cytotoxic drugs of both cancerous and healthy cells, inducing side effects to every rapidly renewing tissue;
    \item \textbf{hormone therapy}, through which the balance of specific hormones gets alterated, which can lead to some side effects such as joint pain osteoporosis;
    \item \textbf{target therapy}, i.e. drugs containing antibodies or inhibitory substances which target the cancer cell and promote its destruction by the immunitary system; target therapy may be difficoult to develop depending on the structure or the function of the target, it may induce unwanted side effects to various organs, and cancer cells can become increasingly resistant to this type of therapy if they find a way to develop which does not depend on the therapy's target \cite{target_therapy1} \todo{\href{https://www.cancer.org/cancer/managing-cancer/treatment-types/targeted-therapy/what-is.html}{check this out}}.
\end{itemize}

In particular, in recent years target therapy has been subject of lots of research, because it could be a powerful mean able to affect only the desired target, helping to reduce the side effects which currently characterize every available cancer cure, possibly limiting the damage to healthy cells \cite{target_therapy3}. \todo{expand target therapy on how it works? if yes, make subsection}

\section{Mutations}

\subsection{Cell signaling and signaling pathways}

\textbf{Cell signaling} is the process through which cells interact with themselves, other cells or their environment; through cells signaling signals are trunsducted, and they can be of many types, usually chemical but also pressure, voltage, temperature or light signals \cite{cell_signaling}. \textbf{Pathways} are a series of actions between molecules inside a cell which lead to a chenge in the cell or the creation of some product \cite{pathway}. Pathways have a \textit{direction} in which the actions happen, and the terms \textit{upstream} and \textit{downstream} are used to indicate whate happens at their beginning or their end respectively. For cancer study, of particular interest are \textbf{signaling pathways}, which allow the transduction of cell signals, because locate and block pathways that are responsible for the core functions of cancer growth, could terminate the development of the latter. \todo{\href{https://www.ncbi.nlm.nih.gov/pmc/articles/PMC8002322/}{check this out}, also check if what i wrote is actually true, i think i read it somewhere but can't find the source right now; expand on cell signaling? expand of pathways? if yes, make subsections}

\subsection{Passenger and driver mutations}

There are two types of mutations in cancer: \textbf{passenger} and \textbf{driver} mutations. Passenger mutations don't provide direct benefits to tumor growth and development, while driver mutations are able to directly influence cancer, providing evolutionary advantage and allowing increase in the number of tumor cells. A \textbf{Driver gene} is a gene which contains at least one driver mutation, but it can also contain passenger mutations \todo{\href{https://www.aiom.it/wp-content/uploads/2019/02/20190524RM_21_Tommasi.pdf}{DO I ADD THIS A CITATION?}}; a \textbf{driver pathway} is made up of at least one driver gene. Driver mutations, genes and pathways are of great scientific interest since hold an important role in cancer proliferation.

Driver genes can be classified in 12 signaling pathways, which regulate functions of survival, fate and genomic manteinance of the cell \todo{use (and expand) this? same source as prev}.

\subsection{Classifying mutations}

\subsection{Mutual exclusivity}

\subsection{\textit{De novo} and \textit{knowledge-based} approaches}

\cleardoublepage
