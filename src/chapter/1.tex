\chapter{Introduction} \label{chap:introduction}

placeholder. \todo{abstract di tutta la tesi}

\section{Cancer}

\subsection{TODO}

placeholder. \todo{introduzione sul cancro}

% Cancer is a medical condition characterized by uncontrolled cell proliferation, which allows cells to infiltrate into organs and tissues, thereby altering their functions and structure. This exponential growth is driven by mutations in cellular DNA, which encodes the instructions for cell development and multiplication, therefore errors in these instructions can lead to cancerous transformation. \todo{expand/rework this; also, find source of this paragraph because i seem to have lost it}

\subsection{Causes}

The development of cancer is a complex, \textit{multistep process} influenced by various factors, making it too simplistic to attribute cancer to a single cause. Nonetheless, many agents, such as radiation, chemicals, and viruses, have been found to induce cancer.

Radiation and many chemical carcinogens work by damaging DNA and \textit{causing mutations}. These are known as \textbf{initiating agents} because they trigger genetic changes that lead to cancer. For example, solar ultraviolet radiation, chemicals in tobacco smoke, and \href{https://en.wikipedia.org/wiki/Aflatoxin}{\textit{aflatoxin}} are well-documented carcinogens. Tobacco smoke, in particular, is a major cause of lung cancer and is also linked to cancers of the oral cavity, throat, larynx, esophagus, and other areas. It is estimated that smoking contributes to a significant portion of all cancer deaths.

In contrast, some carcinogens --- known as \textbf{tumor promoters} --- facilitate cancer development by \textit{stimulating cell proliferation} rather than by inducing mutations. Tumor formation in animal models typically require both an initiating agent and a promoter to facilitate the growth of mutated cells. For instance, hormones (especially estrogens) play a role as tumor promoters in certain cancers.

Additionally, some viruses are known to cause cancer in both animals and humans, such as those linked to liver cancer and cervical carcinoma. These viral-induced cancers highlight the broader impact of carcinogens and underscore their role in both viral and non-viral cancer development \cite{nih_cancer_dev}.

In summary, the various ways in which different factors contribute to cancer emphasize the complexity of the disease and underscore the importance of developing effective treatment approaches, which will be explored in later sections.

\subsection{Mutations in cancer development}

The fundamental feature of cancer development is \textbf{tumor clonality}, meaning tumors often develop from single cells that start to proliferate abnormally. However, the clonal origin of tumors does not mean that the initial progenitor cell had all the features of a cancer cell from the start. Instead, cancer evolves through a multistep process in which cells \textit{gradually acquire malignant characteristics} through a series of \textbf{alterations}. This multistep nature is indicated by the fact that most cancers develop later in life. For example, the incidence of colon cancer increases markedly with age, showing a dramatic rise as individuals grow older. This steep age-related increase suggests that cancer typically results from \textbf{multiple abnormalities} accumulated over many years.

At the cellular level, cancer development is viewed as a process of mutation and selection for cells with progressively greater abilities to proliferate, survive, invade, and metastasize. The first stage, known as \textbf{tumor initiation}, involves a genetic alteration that triggers abnormal growth in a single cell, leading to the expansion of a population of clonally derived tumor cells. \textbf{Tumor progression}, continues as \textit{additional mutations} arise within this cell population, with some mutations providing a selective advantage. As a result, cells bearing these advantageous mutations become \textit{dominant} within the tumor, a process known as \textbf{clonal selection}. This selection continues throughout the tumor's evolution, causing it to grow more rapidly and become increasingly malignant \cite{nih_cancer_dev}.

Undoubtedly, mutations are fundamental to the development of cancer and to its progression. Therefore, to effectively combat this disease, it is essential to gain a comprehensive understanding of how these genetic alterations occur and contribute to tumor development.

\section{Targeted therapy}

\subsection{Current cancer treatment}

Research aimed at finding cancer treatment is continuously evolving due to the disease's lethality and complexity. Currently, the primary techniques used to remove, control, manage, and delay the effects of cancer include \cite{cancer_treat}:

\begin{itemize}
    \item \textit{surgery}, which involves the removal of the cancerous region and is generally reserved for solid tumors;
    \item \textit{radiotherapy}, which uses x-rays to destroy tumor cells, aiming to target the cancerous region as precisely as possible to preserve healthy tissue; however, radiotherapy can increase the risk of developing secondary tumors, such as leukemia or sarcomas, and may lead to delayed effects like dementia, amnesia, or progressive cognitive difficulties;
    \item \textit{chemotherapy}, which employs \textit{cytotoxic} drugs to block cellular division in both cancerous and healthy cells, but they can also induce side effects in rapidly renewing tissues;
    \item \textit{hormone therapy}, which alters the balance of specific hormones, potentially leading to side effects such as joint pain or osteoporosis.
\end{itemize}

Recent advancements in traditional cancer treatments like chemotherapy, radiotherapy, and surgery have contributed to a decline in cancer mortality rates over the years. However, these methods still face significant limitations, often resulting in tumor recurrence and mortality, due to their various side effects. This has prompted a shift toward \textbf{mutation-targeted therapies}, as a result of their potential to precisely target cancer cells and minimize damage to healthy cells and tissue \cite{target_therapy1, jci}.

\subsection{Overview and origin}

\textbf{Targeted therapy} is a form of cancer treatment that targets proteins responsible for the growth, division, and spread of cancer cells, and it forms the basis of \href{https://en.wikipedia.org/wiki/Personalized_medicine}{precision medicine}. The targets include growth factor receptors, signaling molecules, cell-cycle proteins, and other molecules crucial for normal tissue development and homeostasis, which often become overexpressed or altered in cancer cells, leading to their aberrant function \cite{se_tt}.

Unlike standard chemotherapy, which indiscriminately destroys both rapidly dividing cancerous and normal cells, targeted therapies specifically attack abnormal proteins produced by mutated genes. Because normal cells lack these tumor-specific mutations, targeted therapies often show a higher degree of selectivity, causing fewer off-target effects and achieving more rapid and substantial tumor reduction \cite{jci}.

The concept of targeted therapy originates from the german Nobel Prize Paul Ehrlich's idea of a \curlyquotes{\textit{magic bullet}} \cite{ehrlich}, when he envisioned a chemical capable of specifically targeting microorganisms. Over a century later, advances in molecular biology enhanced our understanding of the mechanisms behind cancer initiation, promotion, and progression. This progress led to the development of treatments that can interfere with specific molecular targets, typically proteins, linked to tumor growth and progression \cite{se_tt}.

\subsection{Therapy types}

Most types of targeted therapy consist of \textbf{small-molecule drugs}, which are used for targets located inside cells because their small size allows them to enter cells easily, and \textbf{monoclonal antibodies}, which are laboratory-produced proteins engineered to bind to specific targets on cancer cells. Some monoclonal antibodies help the immune system identify and destroy cancer cells by marking them, while others directly inhibit the growth of cancer cells or induce their self-destruction, and still others deliver toxins directly to cancer cells \cite{target_therapy1}.

Most targeted therapies treat cancer by interfering with specific proteins that promote tumor growth and spread. This approach differs from chemotherapy, which often kills all rapidly dividing cells. The following are the different approaches that targeted therapy employs \cite{target_therapy1}.

\begin{itemize}
    \item \textit{Immunotherapy}. Cancer cells can often evade detection by the immune system. Certain targeted therapies mark cancer cells, making them easier for the immune system to identify and destroy, while others enhance the immune system's ability to fight cancer more effectively.
    \item \textit{Signal interruption}. Targeted therapies can interrupt signals that cause cancer cells to grow and divide uncontrollably. Cells normally divide in response to specific signals binding to proteins on their surface. However, some cancer cells present changes in the proteins that tell them to divide without the signals. Targeted therapies can block these proteins, slowing the uncontrolled growth of cancer.
    \item \textit{Angiogenesis inhibition}. The process through which new blood vessels form is called \href{https://en.wikipedia.org/wiki/Angiogenesis}{angiogenesis}; beyond a certain size tumors need new blood vessels, thus the tumor sends signals to start angiogenesis. Some targeted therapies can disrupt the signals that trigger this process, preventing the formation of a blood supply, and restricting the tumor's size. 
    \item \textit{Cell-killing agents delivery}. Some monoclonal antibodies are combined with substances like toxins, chemotherapy drugs, or radiation. These antibodies bind to targets on the surface of cancer cells, delivering the cell-killing agents directly into the cells, causing them to die. Most importantly, cells without these targets remain unharmed.
    \item \textit{Apoptosis activation}. Cancer cells often evade the natural process of cell death, known as \href{https://en.wikipedia.org/wiki/Apoptosis}{apoptosis}, which initiates when cells become damaged or are no longer needed. Some targeted therapies can trigger apoptosis in cancer cells, leading to their death.
    \item \textit{Hormone therapy}. Some types of breast and prostate cancer require specific hormones to grow. Hormone therapies block the body's production of growth hormones or prevent them from acting on cells, including cancer cells.
\end{itemize}

The diverse strategies employed by targeted therapies highlight the innovative approaches being developed to treat cancer more precisely. As research advances, these methods will continue to evolve, potentially improving outcomes and reducing side effects compared to traditional treatments.

\subsection{Drawbacks and side effects}

Like all cancer treatment, targeted therapy also has limitations, and often works best when combined with other types of targeted therapies or additional cancer treatments like chemotherapy and radiation \cite{target_therapy1}.

In particular, developing drugs for certain targets can be challenging due to factors including the target's structural complexity, its function within the cell, or a combination of both. Moreover, cancer cells can develop resistance to targeted therapy, which may occur if the target itself mutates, rendering the therapy unable to interact with it effectively. Alternatively, resistance can arise if cancer cells adapt and find new growth mechanisms that do not rely on the target \cite{target_therapy1}.

% \todo{ho un paper con il quale è possibile espandere questa sezione ma ho tolto la sezione che avevo scritto perché non credo sia necessario (almeno per il momento) scendere così tanto nel dettaglio di questa parte}

% The following are several ways in which cancer cells can develop resistance to targeted therapy \cite{target_therapy2}.
%
% \paragraph{Direct target reactivation}
%
% One of the first and most notable successes in targeted cancer therapy was the use of \href{https://en.wikipedia.org/wiki/Imatinib}{\textit{imatinib}}, a \textit{small-molecule drug} which inhibits ABL-kinase, employed to treat chronic myelogenous leukemianitial. Studies of patients who relapsed despite imatinib treatment revealed that BCR-ABL kinase activity often reappeared. This discovery revealed a key mechanism of resistance to targeted therapy: the direct restoration of the biological function that was previously disrupted by \textit{imatinib}.
%
% Another instance is the use of potent anti-androgens such as \textit{enzalutamide} can effectively control disease even in CRPC (\textit{Castration-Resistant prostate cancer}). Notably, a prevalent mechanism of resistance to \textit{enzalutamide} involves the removal of the drug-binding domain of the androgen receptor through \href{https://en.wikipedia.org/wiki/Alternative_splicing}{alternative splicing}. Indeed, alternate splicing has been frequently observed as a method for reactivating targets in response to targeted inhibitors.
%
% Another form of on-target resistance involves the increased expression of the targeted oncoprotein through transcriptional upregulation or genomic amplification. This has been noted in melanoma, NSCLC (\textit{Non-small Cell Lung Cancer}), and prostate cancer, and can be counteracted with stronger inhibitors and blockade of downstream signaling.
%
% These examples show that restoring the biological function of targeted oncoproteins is a key mechanism by which cancer cells evade targeted therapies and overcome \href{https://en.wikipedia.org/wiki/Oncogene_addiction}{oncogene addiction}.
%
% \paragraph{Oncogenic pathway signals activation}
%
% \paragraph{Alternative oncogenic pathways engagement}
%
% \paragraph{Adaptive survival mechanisms}

As for side effects, in general, targeted molecular therapies have good toxicity profiles. However, side effects differ from person to person, even among those undergoing the same cancer treatment \cite{nih_se}, and some patients may be highly sensitive to these drugs and may develop specific and severe toxicities \cite{se_tt}.

The most common side effects of targeted therapy are diarrhea and liver issues, but they may also include problems with blood clotting and wound healing, high blood pressure, fatigue, mouth sores, nail changes, loss of hair color, and skin problems. In rare cases, a perforation may occur in the wall of the esophagus, stomach, small intestine, colon, rectum, or gallbladder. Medications are available to manage many of these side effects, either by preventing them or treating them once they arise. Additionally, most side effects of targeted therapy subside after the treatment is completed \cite{target_therapy1}.

In conclusion, although targeted therapy shows promise with generally manageable side effects, it has limitations such as potential drug resistance and varying individual responses. Effective management of these side effects and ongoing research are essential to improving treatment outcomes and patient care.

% \paragraph{DAEs}
%
% Targeted therapies and immunotherapies are linked to a variety of \textit{Dermatologic Adverse Events} (DAEs) due to the involvement of common signaling pathways in both malignant processes and the normal functions of the epidermis and dermis. Dermatologic toxicities can affect the skin, oral mucosa, hair, and nails.
%
% The most frequent DAE is \textit{acneiform rash}, seen in multiple of patients undergoing treatment that include epidermal growth factor receptor (EGFR) inhibitors. BRAF inhibitors are known to cause secondary skin tumors, including \textit{squamous cell carcinoma} and \textit{keratoacanthomas}, and can also alter pre-existing pigmented lesions, and induce hand-foot skin reactions. (\textit{Immune checkpoint inhibitors} (ICIs) typically lead to nonspecific \textit{maculopapular rash}, but can also cause \textit{psoriatic} lesions, \textit{lichenoid dermatitis}, \textit{xerosis}, and \textit{pruritus}.
%
% Among the oral mucosal toxicities from targeted therapies, \textit{oral mucositis} is most frequent with mammalian Target of Rapamycin (mTOR) inhibitors, followed by \textit{stomatitis} from multikinase angiogenesis and HER inhibitors, geographic tongue, and \textit{hyperpigmentation}. ICIs generally cause oral lichenoid reactions and \textit{xerostomia}.
%
% Regarding skin problems, \textit{alopecia} is a frequent side effect of both targeted and endocrine therapies. Moreover, targeted therapies can damage the nail folds, resulting in \textit{paronychia} and \textit{periungual pyogenic granuloma}. Mild \textit{onycholysis}, brittle nails, and slower nail growth may also occur \cite{skin_nails}.
%
% \paragraph{Gastrointestinal toxicity}
%
% Diarrhea serves as a dose-limiting toxicity for many small-molecule tyrosine kinase inhibitors. \href{https://en.wikipedia.org/wiki/Gefitinib}{\textit{Gefitinib}}, in particular, has been associated with significant mucosal toxicity when administered at higher doses, although the exact mechanism remains unclear. Research indicates that many patients with colorectal cancer treated with \textit{gefitinib} alongside \textit{irinotecan}, \textit{5-fluorouracil}, and \textit{leucovorin} developed a gastrointestinal syndrome characterized by abdominal pain and diarrhea, necessitating dose reductions and preventing further development of this combination \cite{se_tt}.
%
% \paragraph{Lung problems}
%
% \textit{Interstitial lung disease} (ILD) is recognized as a potential adverse effect of certain cancer chemotherapeutic agents and local radiotherapy. Recently, ILD related to \textit{gefitinib} has emerged as a significant toxicity concern. Studies reported that multiple patients with \textit{non-small cell lung cancer} (NSCLC), treated with \textit{gefitinib}, developed ILD, and the condition led to a fatal outcome in some of them. Although the exact mechanism of \textit{gefitinib}-related ILD is not fully understood, EGFR is believed to play a crucial role in maintaining and repairing epithelial tissues and regulating mucin production in the airways. \textit{Gefitinib} appears to primarily induce pulmonary side effects in patients with existing pulmonary conditions \cite{se_tt}.

\subsection{Drugs targeting mutations}

As mentioned earlier, mutations play a crucial role in the growth and development of cancer. Targeted therapy allows for precise targeting of the mutations that enable cancer to continue its progression. In particular, oncogenic gene mutations may be druggable in several ways \cite{jci}:

\begin{itemize}
    \item some oncogenic gene mutations encode proteins that are structurally or functionally different from the wild-type (WT), normal version of the protein; these differences create an opportunity for developing targeted therapies, because a drug can be designed specifically to bind to these unique features, and inhibit the protein's activity, without affecting the WT protein in healthy cells;
    \item gene mutations often result in the abnormal activation of some protein, through mechanisms like a \textit{gain-of-function mutation} or \textit{gene amplification}; although these proteins are considered druggable, the mutation does not necessarily change the protein in a way that allows for mutant-specific targeting, i.e. drugs may also target the WT version of the protein present in healthy cells, potentially leading to more side effects;
    \item some oncogenic mutations create novel molecular dependencies or vulnerabilities in cancer cells, which can be exploited by targeted therapies; these are called \textit{actionable mutations} because they provide new targets for drug development that are specific to cancer cells and do not exist in normal cells.
\end{itemize}

While truly druggable mutations in the first category are relatively rare, many overactive or amplified targets still offer effective therapeutic opportunities due to their elevated expression levels or the significant dependence of cancer cells on these specific proteins. Additionally, mutations that currently lack targeted therapy options can still function as biomarkers to guide other therapeutic decisions \cite{jci}.

Advances in targeted therapies have been significantly driven by technological progress in sequencing over the past two decades, particularly with the development of \href{https://en.wikipedia.org/wiki/Massive_parallel_sequencing}{next-generation sequencing} (NGS). The identification of both common and rare genetic mutations has launched research into targeted therapies against mutant proteins and aberrant molecular signaling pathways. Moreover, the discovery of the \href{https://en.wikipedia.org/wiki/Philadelphia_chromosome}{BCR-ABL fusion gene} and the development of the BCR-ABL inhibitor \href{https://en.wikipedia.org/wiki/Imatinib}{\textit{imatinib}} marked a breakthrough in targeted cancer therapies, leading to numerous FDA-approved drugs \cite{jci}. However, the challenge of developing targeted therapies remains difficult, particularly for mutations that affect normal and cancerous proteins alike, or those for which no targeted therapies currently exist. The complexities of druggable mutations and their effects on treatment underscore the need for ongoing research and refinement in this area. Given the importance of fully understanding the role of mutations in cancer development, in order to improve targeted therapies and cancer treatment overall, research must focus on genomic mutations and their classification. The next chapter will discuss the existence of different types of mutations and the current techniques used to classify them.

\cleardoublepage
