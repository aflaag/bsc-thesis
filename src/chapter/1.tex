\chapter{Introduction} \label{chap:introduction}

Cancer is a medical condition characterized by uncontrolled cell proliferation, which allows cells to infiltrate into organs and tissues, thereby altering their functions and structure. This exponential growth is driven by mutations in cellular DNA, which encodes the instructions for cell development and multiplication, therefore errors in these instructions can lead to cancerous transformation. In most types of cancer, a single aberration is insufficient for cancer development; instead, multiple mutations are required. Some of these mutations are present since birth, while others occur throughout life due to chance or lifestyle choices. Additionally, for tumor proliferation to occur, mutations in genes that regulate cell growth are necessary \cite{Vogelstein2004}. Specifically, proto-oncogenes, which promote mitosis, and tumor suppressor genes, which inhibit cell growth, are involved in this process, known as \textit{oncoevolution} \cite{carcinogenesis}. \todo{expand this introduction}

\section{Cancer therapy}

\subsection{Current treatment}

Research aimed at finding a cure for cancer is continuously evolving due to the tumor's lethality and complexity. Currently, the primary techniques used to remove, control, manage, and delay the effects of cancer include \cite{cancer_treat}:

\begin{itemize}
    \item \textbf{surgery}, which involves the removal of the cancerous region and is generally reserved for solid tumors;
    \item \textbf{radiotherapy}, which uses x-rays to destroy tumor cells, aiming to target the cancerous region as precisely as possible to preserve healthy tissue; however, radiotherapy can increase the risk of developing secondary tumors, such as leukemia or sarcomas, and may lead to delayed effects like dementia, amnesia, or progressive cognitive difficulties;
    \item \textbf{chemotherapy}, which employs cytotoxic drugs to block cellular division in both cancerous and healthy cells, but they can also induce side effects in rapidly renewing tissues.
    \item \textbf{hormone therapy}, which alters the balance of specific hormones, potentially leading to side effects such as joint pain or osteoporosis;
    \item \textbf{targeted therapy}, which involves drugs containing antibodies or inhibitory substances that specifically target cancer cells, promoting their destruction by the immune system; however, developing effective targeted therapies can be challenging due to the complexity of the target's structure or function; in addition, this approach may also induce unwanted side effects in various organs, and cancer cells may develop resistance if they find alternative ways to develop that do not rely on the therapy's target \cite{target_therapy1} \todo{\href{https://www.cancer.org/cancer/managing-cancer/treatment-types/targeted-therapy/what-is.html}{check this out}}. \end{itemize}

\subsection{Target therapy}

In particular, in recent years targeted therapy has been the focus of extensive research due to its potential to precisely affect only the desired target, thereby reducing the side effects that currently characterize most cancer treatments and potentially limiting damage to healthy cells \cite{target_therapy3}. \todo{expand target therapy on how it works; check README for link}

\cleardoublepage
