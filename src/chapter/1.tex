\chapter{Introduction} \label{chap:introduction}

\section{Context}

\subsection{Cancer}

Cancer is a medical condition characterized by uncontrolled cell proliferation, which allows cells to infiltrate into organs and tissues, thereby altering their functions and structure. This exponential growth is driven by mutations in cellular DNA, which encodes the instructions for cell development and multiplication, therefore errors in these instructions can lead to cancerous transformation. In most types of cancer, a single aberration is insufficient for cancer development; instead, multiple mutations are required. Some of these mutations are present since birth, while others occur throughout life due to chance or lifestyle choices. Additionally, for tumor proliferation to occur, mutations in genes that regulate cell growth are necessary \cite{Vogelstein2004}. Specifically, proto-oncogenes, which promote mitosis, and tumor suppressor genes, which inhibit cell growth, are involved in this process, known as \textit{oncoevolution} \cite{carcinogenesis}. \todo{talk about oncogenes in depth?}

\subsection{Cancer treatment}

Research aimed at finding a cure for cancer is continuously evolving due to the tumor's lethality and complexity. Currently, the primary techniques used to remove, control, manage, and delay the effects of cancer include \cite{cancer_treat}:

\begin{itemize}
    \item \textbf{surgery}, which involves the removal of the cancerous region and is generally reserved for solid tumors;
    \item \textbf{radiotherapy}, which uses x-rays to destroy tumor cells, aiming to target the cancerous region as precisely as possible to preserve healthy tissue; however, radiotherapy can increase the risk of developing secondary tumors, such as leukemia or sarcomas, and may lead to delayed effects like dementia, amnesia, or progressive cognitive difficulties;
    \item \textbf{chemotherapy}, which employs cytotoxic drugs to block cellular division in both cancerous and healthy cells, but they can also induce side effects in rapidly renewing tissues.
    \item \textbf{hormone therapy}, which alters the balance of specific hormones, potentially leading to side effects such as joint pain or osteoporosis;
    \item \textbf{targeted therapy}, which involves drugs containing antibodies or inhibitory substances that specifically target cancer cells, promoting their destruction by the immune system; however, developing effective targeted therapies can be challenging due to the complexity of the target's structure or function; in addition, this approach may also induce unwanted side effects in various organs, and cancer cells may develop resistance if they find alternative ways to develop that do not rely on the therapy's target \cite{target_therapy1} \todo{\href{https://www.cancer.org/cancer/managing-cancer/treatment-types/targeted-therapy/what-is.html}{check this out}}. \end{itemize}

In recent years, targeted therapy in particular has been the focus of extensive research due to its potential to precisely affect only the desired target, thereby reducing the side effects that currently characterize most cancer treatments and potentially limiting damage to healthy cells \cite{target_therapy3}. \todo{expand target therapy on how it works? if yes, make subsection}

\section{Mutations}

\subsection{Cell signaling and signaling pathways}

\textbf{Cell signaling} is the process by which cells interact with each other, themselves, or their environment. This involves the transduction of signals, which can be chemical, or can involve other types such as pressure, temperature, or light signals \cite{cell_signaling}. \textbf{Pathways} are sequences of molecular interactions within a cell that lead to a change in the cell or the production of a specific product \cite{pathway}. These pathways have a direction in which the actions occur, with the terms \textit{upstream} and \textit{downstream} indicating the initial and final stages of these processes, respectively.

In cancer research, signaling pathways are of particular interest because they mediate the transduction of cell signals. Identifying and targeting the signaling pathways responsible for cancer growth could potentially halt the development of the disease. \todo{\href{https://www.ncbi.nlm.nih.gov/pmc/articles/PMC8002322/}{check this out}, also check if what i wrote is actually true, i think i read it somewhere but can't find the source right now; expand on cell signaling? expand of pathways? if yes, make subsections}

\subsection{Passenger and driver mutations}

There are two types of mutations in cancer: \textbf{passenger mutations} and \textbf{driver mutations}. Passenger mutations do not confer direct benefits to tumor growth or development, whereas driver mutations actively contribute to cancer progression by providing an evolutionary advantage and promoting the proliferation of tumor cells. A \textbf{driver gene} is a gene that harbors at least one driver mutation, though it may also contain passenger mutations \todo{\href{https://www.aiom.it/wp-content/uploads/2019/02/20190524RM_21_Tommasi.pdf}{DO I ADD THIS AS A CITATION???}}. A driver pathway consists of at least one driver gene. Driver mutations, genes, and pathways are of significant scientific interest due to their crucial role in cancer proliferation.

Driver genes can be classified into 12 signaling pathways, which regulate cellular functions related to survival, fate, and genomic maintenance. \todo{use (and expand) this? same source as prev}

\section{Classifying mutations}

\subsection{Frequency}

To classify mutations into the two categories described, assessing their biological function is essential, though this remains a challenging task. Numerous methods exist to predict the functional impact of mutations based on \textit{a priori} knowledge. However, these approaches often fail to integrate information effectively across various mutation types and are limited by their reliance on known proteins, rendering them less effective for less-studied ones.

With the decreasing cost of DNA sequencing, it is now feasible to categorize mutations by examining their frequency, as driver mutations are typically the most recurrent in patients' genomes. However, this approach often fails because driver mutations can vary significantly between patient samples, even within the same cancer type, which reduces the statistical power of frequency analyses \cite{multi-dendrix}. This heterogeneity is largely due to driver mutations being predominantly located in genes that are part of cell signaling pathways, leading to different patients harboring mutations in different pathway loci. Therefore, studies should be conducted at the pathway level, as it is well established that different mutations can affect the same pathway across multiple samples \cite{multi-dendrix}. However, since each pathway involves multiple genes, numerous possible combinations of driver mutations could impact a crucial cancer pathway, making it computationally unfeasible to test every possible gene permutation \cite{dendrix} --- estimates suggest that the human genome contains more than 50,000 genes \cite{n-genes}. Additionally, the minimal overlap of mutated genes across sample pairs, even from the same patient \cite{mdpfinder}, along with the heterogeneity discussed, complicates the categorization of mutations based on frequency since demonstrating the recurrence of rare mutations necessitates a very large number of patients \cite{dendrix}. Therefore, studies should be conducted at the pathway level, as it is well established that different mutations can affect the same pathway across multiple samples \cite{multi-dendrix}.

\subsection{Mutual exclusivity and coverage}

Most techniques developed in recent years for recognizing driver mutations leverage a statistical property observed in cancer patient data: each patient typically has a relatively small number of mutations that affect multiple pathways, thus each pathway will contain \textit{1 driver mutation on average} per sample. This concept of mutual exclusivity among driver mutations within the same pathway, as statistically observed in patient samples, is then axiomatized and employed by research algorithms designed to identify driver mutations \cite{multi-dendrix}. Additionally, mutual exclusivity \textit{does not affect different pathways}; it is a phenomenon that occurs exclusively within a single pathway. While the precise explanation for this occurrence is not yet fully understood, several hypotheses appear promising \cite{survey, mutual_exclusivity_expls}.

\begin{itemize}
    \item one hypothesis is that mutually exclusive genes are functionally connected within a common pathway, acting on the same downstream effectors and creating functional redundancy; consequently, they would share the same selective advantage, meaning that the alteration of one mutually exclusive gene would be sufficient to disrupt their shared pathway, thereby removing the selective pressure to alter the others; this explanation, however, does not fully account for the phenomenon because the co-alteration of mutually exclusive genes should not result in negative effects on the cell.
    \item an alternative explanation is that the co-occurrence of mutually exclusive alterations is detrimental to cancer survival, leading to the elimination of cells that harbor such co-occurrences; moreover, some pairs of mutually exclusive genes could be \textit{synthetic lethal}, meaning that while the alteration of one gene may be compatible with cell survival, the simultaneous aberration of both genes would be lethal to the cell \todo{add example from survey paper?; also, use \href{https://www.nature.com/articles/s41467-020-20820-x}{example}? (mail "Risposte (parziali) alle questioni, ERG e SPOP")}.
\end{itemize}

In addition, another key property of driver pathways is \textbf{coverage}, i.e. driver genes constituting a driver pathway are frequently mutated across many samples. Thus, \textit{a driver pathway consists of genes that are mutated in numerous patients, with mutations being approximately mutually exclusive}. It is also observed that pathways exhibiting these characteristics are generally shorter and comprised of fewer genes on average \cite{multi-dendrix}.

\subsection{\textit{De novo} and \textit{knowledge-based} approaches}

Despite not knowing the true explanation for mutual exclusivity yet, and not knowing whether it holds therapeutical potential, it is a very recurrently observed behavior in data, and many believe that it may lead to some discoveries for cancer treatment. The existing approaches can be categorized into two types: \textit{de novo} approaches, which perform the research of mutually exclusive patterns by using only genomic data from patients, and \textit{knowledge-based} methods, which integrate the analysis with external \textit{a priori} information \cite{survey}. Evidently, \textit{de novo} approaches may not have enough information because they do not draw on existing databases; on the counter side, the current understanding of gene and protein interactions in humans is still incomplete, and most existing pathway databases fail to accurately represent the specific pathways and interactions present in cancer cells. As a result, \textit{de novo} methods may find new but not necessarily accurate results, while \textit{knowledge-based} approaches' excessive focus on known data sources may constrain opportunities for uncovering new biological insights \cite{multi-dendrix}.

Although the true explanation for mutual exclusivity remains unknown, and its therapeutic potential is still uncertain, this phenomenon is frequently observed in data and is thought to potentially lead to discoveries in cancer treatment. Existing approaches can be categorized into two types: \textit{de novo} approaches, which identify mutually exclusive patterns using only genomic data from patients, and \textit{knowledge-based} methods, which integrate the analysis with external \textit{a priori} information \cite{survey}. De novo approaches might lack sufficient information as they do not utilize existing databases. Conversely, given that our understanding of gene and protein interactions in humans is still incomplete and many pathway databases fail to accurately represent the specific pathways and interactions present in cancer cells, \textit{knowledge-based} approaches may be limited by their dependence on existing data sources. Consequently, \textit{de novo} methods might yield new but potentially less accurate results, while \textit{knowledge-based} approaches may limit the discovery of novel biological insights \cite{multi-dendrix}.

\cleardoublepage
