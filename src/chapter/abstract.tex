\begin{abstract}
    Cancer is one of the leading causes of death globally, responsible for millions of deaths each year. It is a complex group of diseases characterized by the uncontrolled growth and spread of abnormal cells in the body, driven by a multitude of genetic mutations. These cells can invade surrounding tissues and, if left untreated, may spread to other parts of the body. Therefore, it is essential to search for an effective treatment, which not only alleviates symptoms and improves quality of life, but also addresses the underlying mechanisms of cancer.

    Over the years, various strategies have been developed and implemented to treat cancer as effectively as possible. However, most currently employed cancer treatments have significant side effects, which can severely affect patients' quality of life. As a result, there is a pressing need for therapies that are not only effective in targeting cancer cells, but also minimize harm to healthy tissues, thereby improving the patient's overall experience during treatment.

    This work will focus specifically on one type of treatment: targeted therapies, which have shown significant results in recent years, both in terms of reducing side effects and improving treatment efficacy. Targeted therapies, as the name suggests, are designed to focus on specific molecules or pathways involved in cancer growth and progression. By precisely targeting these elements, targeted therapies aim to disrupt the cancer's ability to proliferate, while minimizing damage to healthy cells, resulting in improved treatment outcomes and fewer side effects.

    However, identifying effective targets is not straightforward, as it depends on a variety of factors. In fact, throughout the years, numerous studies have attempted to differentiate between the vast number of mutations that occur in cancer, aiming to identify the most relevant ones for tumor treatment.

    Fortunately, certain phenomena in genomic data have been identified that could potentially be leveraged to simplify the search in this field. This work will explore the computational approaches adopted in recent years, the challenges this search presents, and the genomic characteristics that newly developed algorithms leverage to classify mutations efficiently.

    The first chapter of this work will provide an overview of cancer, exploring its causes, current treatment options, and a more detailed examination of targeted therapies. The second chapter will highlight the challenges associated with differentiating among the vast number of mutations involved in cancer, and will introduce the studies that will be analyzed throughout this work, focusing specifically on how they mathematically formulated biological phenomena observed statistically in genomic data. The third chapter will delve deeper into the algorithms employed by the discussed studies, and explain how they employed their own metrics. Finally, the last chapter will address additional considerations regarding the studies presented.
\end{abstract}

\let\cleardoublepage\clearpage
