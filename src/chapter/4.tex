\chapter{Discussion} \label{chap:discussion}

placeholder. \todo{introduction to the chapter}

\section{Dendrix}

\subsection{The deterministic formalization}

\textcite{dendrix} provided one of the first mathematical formalizations of the phenomena of mutual exclusivity and coverage, in the context of gene mutations. Specifically, the definitions introduced offer a very intuitive approach to formalizing these biological concepts:

\begin{itemize}
    \item the coverage of a gene is defined as the set of patients exhibiting a mutation of the gene, equivalent to the number of 1s in its column of the mutation matrix;
    \item a set of genes is defined to be mutually exclusive if no patient has more than one mutated gene in the set, i.e. no row of the set's associated matrix has more than 1 one;
    \item the coverage of a gene set is the set of patients with at least one mutation in the set;
    \item the coverage overlap of a gene set is the count of patients who possess more than one mutation within the gene set;
    \item the weight of a gene set is calculated as the difference between the coverage of the gene set and its coverage overlap.
\end{itemize}

Consequently, a higher weight for a gene set indicates both greater coverage and mutual exclusivity among its genes. The weight formula suggests that the optimal gene set, i.e. the one that maximazes its weight, is the one where the associated matrix has a high number of rows with at least one 1, and a minimal number of rows with more than one 1.

In my view, this metric stands out as the most elegant among those discussed in this work: it not only provides a clear and intuitive measure, but also offers a simple and straightforward formula. While this formula may seem to oversimplify the challenge of identifying driver pathways --- given that mutual exclusivity alone does not cover all aspects of pathway analysis, and exact mutual exclusivity is rarely observed in real data --- it remains a highly regarded deterministic formalization, and numerous studies (some of which are discussed in this work) agree that this metric represents the most refined approach to date.

\subsection{Additional considerations}

I want to commend \textcite{dendrix} for their precise and methodical explanation of their methodology. With only a few minor, negligible details to consider, their work is exceptionally clear regarding their objectives, the methods employed to achieve them, and their actual outcomes. Additionally, their mathematical analysis is thorough and well-supported: in the supplemental material, they provide extensive proofs, including the NP-hardness of the MWSP (which is described in \cref{mwsp proof}), the correctness of their greedy algorithm, and the rapid mixing property of their MCMC approach. I greatly appreciate the clarity of their presentation, which significantly facilitated my understanding of their study. Indeed, this level of clarity is notably superior compared to other papers I have analyzed, which lack such clear explanations, as will be discussed in the subsequent sections.

As a final note, I believe that conducting a comparative analysis between the MCMC approach and a \href{https://en.wikipedia.org/wiki/Random_search}{random search} method would be interesting. For instance, consider the following algorithm:

\begin{enumerate}
    \item \textit{initialization}: given the set of all genes $\mathcal G$, choose an arbitrary subset $M_0 \subseteq \mathcal G$ ok $k$ genes;
    \item \textit{iteration}: for $t = 1, 2, \ldots$ derive $M_{t + 1}$ from $M_t$ as follows:

    \begin{enumerate}
        \item define $W \subseteq \mathcal G$ and $V \subseteq M_t$ randomly;
        \item choose $\displaystyle (\hat w, \hat v) \in \argmax_{(w, v) \in W \times V}{W\rbk{\rbk{M_t - \{v\}} \cup \{w\}}}$;
        \item set $M_{t + 1} := \rbk{M_t - \{\hat v\}} \cup \{\hat w\}$.
    \end{enumerate}
\end{enumerate}

At each step, this algorithm selects a predetermined amount of \textit{random adjustments}, choosing the one that maximizes the weight as the base set for the next iteration. It would be interesting to evaluate how this approach performs on real data, and whether the MCMC algorithm outperforms it, particularly when applied to data under the GIM model.

\section{Multi-Dendrix}

\subsection{The ILP of Dendrix}

\textcite{multi-dendrix} formulated the MWSP as an ILP, which I believe offers a more intuitive and natural approach to the problem. However, as highlighted by several authors, the set $M$ that maximizes $W(M)$ may not always represent an actual biological driver pathway. This limitation stems from the fact that exact mutual exclusivity and coverage are rarely observed in real mutation data, making exact solutions to the MWSP potentially unrealistic. Therefore, I believe a more statistical approach or a probabilistic method, such as the MCMC algorithm used by \textcite{dendrix}, may offer more reliable results in certain contexts.

\subsection{The ILP of Multi-Dendrix}

The ILP formulation for the simultaneous identification of multiple driver pathways is a natural extension of Dendrix's ILP. However, it may face the same challenge in that optimal solutions might not correspond to actual biological pathways. Moreover, \textcite{multi-dendrix} highlight that while the ILP used in Multi-Dendrix effectively finds optimal solutions, it does not rigorously explore suboptimal solutions, in contrast with the MCMC approach, which samples suboptimal solutions based on their weight.

Additionally, the weight function $W'(M)$ in Multi-Dendrix does not explicitly account for the co-occurrence of mutations between genes in different sets. Instead, it prioritizes gene sets with high coverage and approximate exclusivity, which may lead to co-occurrence due to high coverage alone (e.g., when all gene sets have full coverage). Given that co-occurrence is crucial in large biological pathways, algorithms that optimize for gene sets where mutations frequently co-occur might be more effective in identifying key components of these pathways \cite{multi-dendrix}.

Lastly, I would like to emphasize that the exposition of their ILP was unclear and somewhat imprecise: in particular, while \textcite{multi-dendrix} introduced the MMWSP, the ILP they solved appears to address a slightly different version of the problem. This discrepancy could potentially raise concerns about time complexity, an important consideration given the large scale of data in this field. Moreover, their definitions of some indicator variable sets lacked precision. Nevertheless, it is important to highlight that, according to survey studies \cite{survey}, this approach is not only among the fastest --- thanks to the efficiency of ILP solvers --- but also performs exceptionally well in terms of both precision and recall.

\section{MDPFinder}

\subsection{The ILP of Dendrix}

To solve the MWSP, \textcite{mdpfinder} formulated an ILP that is both identical in formulation and constraints to the one proposed by \textcite{multi-dendrix}. Although the MDPFinder paper was published in 2012 and the Multi-Dendrix paper in 2013, the latter does not reference the work of \textcite{mdpfinder}, and both papers present the ILP formulation of Dendrix as their own innovation.

For clarity and detailed definitions of the indicator variables, I have chosen to attribute the formulation to \textcite{multi-dendrix}. In fact, despite previous criticisms of their MMWSP's ILP exposition, their presentation of MWSP's ILP was clearer compared to that of \textcite{mdpfinder}, which would have been challenging to comprehend without additional explanations.

\subsection{The GA}

I find the use of genetic algorithms particularly appealing as a conceptual approach for optimizing a given score function, especially when a computationally efficient optimization method is not readily available. Indeed, genetic algorithms offer a flexible and intuitive framework for exploration and optimization. However, a significant drawback is their relatively slow performance compared to other methods. As noted by \textcite{mdpfinder}, their genetic algorithm is considerably slower than the alternatives they tested, being \textit{12 times slower} than the MCMC algorithm and \textit{over 60 times slower} than the ILP.

Despite its undeniable slowness, I appreciate the versatility of this approach. It allows for easy modification of the objective function, and offers a well-suited integration procedure that may be challenging to incorporate into other algorithm types, such as the ILP. Additionally, the ability to explore suboptimal solutions is valuable, as shown in many findings reported in this work. Finally, their explanation of the GA is quite comprehensible, and I value their effort to compare the results across three different approaches.

\cleardoublepage
