\chapter{Discussion} \label{chap:discussion}

placeholder. \todo{introduction to the chapter}

\section{Dendrix}

\subsection{The deterministic formalization}

\textcite{dendrix} provided one of the first mathematical formalizations of the phenomena of mutual exclusivity and coverage, in the context of gene mutations. Specifically, the definitions introduced offer a very intuitive approach to formalizing these biological concepts:

\begin{itemize}
    \item the coverage of a gene is defined as the set of patients exhibiting a mutation of the gene, equivalent to the number of 1s in its column of the mutation matrix;
    \item a set of genes is defined to be mutually exclusive if no patient has more than one mutated gene in the set, i.e. no row of the set's associated matrix has more than 1 one;
    \item the coverage of a gene set is the set of patients with at least one mutation in the set;
    \item the coverage overlap of a gene set is the count of patients who possess more than one mutation within the gene set;
    \item the weight of a gene set is calculated as the difference between the coverage of the gene set and its coverage overlap.
\end{itemize}

Consequently, a higher weight for a gene set indicates both greater coverage and mutual exclusivity among its genes. The weight formula suggests that the optimal gene set, i.e. the one that maximazes its weight, is the one where the associated matrix has a high number of rows with at least one 1, and a minimal number of rows with more than one 1.

In my view, this metric stands out as the most elegant among those discussed in this work: it not only provides a clear and intuitive measure, but also offers a simple and straightforward formula. While this formula may seem to oversimplify the challenge of identifying driver pathways --- given that mutual exclusivity alone does not cover all aspects of pathway analysis, and exact mutual exclusivity is rarely observed in real data --- it remains a highly regarded deterministic formalization, and numerous studies (some of which are discussed in this work) agree that this metric represents the most refined approach to date.

\subsection{Additional considerations}

I want to commend \textcite{dendrix} for their precise and methodical explanation of their methodology. With only a few minor, negligible details to consider, their work is exceptionally clear regarding their objectives, the methods employed to achieve them, and their actual outcomes. Additionally, their mathematical analysis is thorough and well-supported: in the supplemental material, they provide extensive proofs, including the NP-hardness of the MWSP (which is described in \cref{mwsp proof}), the correctness of their greedy algorithm, and the rapid mixing property of their MCMC approach. I greatly appreciate the clarity of their presentation, which significantly facilitated my understanding of their study. Indeed, this level of clarity is notably superior compared to other papers I have analyzed, which lack such clear explanations, as will be discussed in the subsequent sections.

As a final note, I believe that conducting a comparative analysis between the MCMC approach and a \href{https://en.wikipedia.org/wiki/Random_search}{random search} method would be interesting. For instance, consider the following algorithm:

\begin{enumerate}
    \item \textit{initialization}: given the set of all genes $\mathcal G$, choose an arbitrary subset $M_0 \subseteq \mathcal G$ ok $k$ genes;
    \item \textit{iteration}: for $t = 1, 2, \ldots$ derive $M_{t + 1}$ from $M_t$ as follows:

    \begin{enumerate}
        \item define $W \subseteq \mathcal G$ and $V \subseteq M_t$ randomly;
        \item choose $\displaystyle (\hat w, \hat v) \in \argmax_{(w, v) \in W \times V}{W\rbk{\rbk{M_t - \{v\}} \cup \{w\}}}$;
        \item set $M_{t + 1} := \rbk{M_t - \{\hat v\}} \cup \{\hat w\}$.
    \end{enumerate}
\end{enumerate}

At each step, this algorithm selects a predetermined amount of \textit{random adjustments}, choosing the one that maximizes the weight as the base set for the next iteration. It would be interesting to evaluate how this approach performs on real data, and whether the MCMC algorithm outperforms it, particularly when applied to data under the GIM model.

\section{Multi-Dendrix}

\subsection{An ILP for the MWSP}

% also, as previously mentioned by the authors of multiple of the paper discussed, the set $M$ that maximizes $W(M)$ may not be a real driver pathway in real life. in fact, i too believe that exact solutions to the associated MWSP problem may not be correct, since 

\cleardoublepage
